\chapter{Einleitung}

Parallel zur Verbreitung des Internets, ist die versendete Datenmenge stark angestiegen. Immer mehr Nutzer wollen auf immer größere werdende Datenmengen möglichst schnell zugreifen. Gleichzeitig ist der Anteil der Nutzer, welche über mobile Endgeräte auf die Daten zugreifen, immer weiter angestiegen. \cite{Enge2019}
Aus diesen Entwicklungen ergeben sich weitere Anforderungen. Zum Einen brauchen vor allem mobile Endnutzer eine schnelle und datensparende Datenübertragung. Zum Anderen werden die benötigten Daten von verschiedenen Geräten angefragt. Dadurch können die benötigten Daten sich von Nutzer zu Nutzer stark unterscheiden.

Diese Anforderungen müssen von der Schnitstelle zwischen Server und Client erfüllt werden. Mit der aktuell wohl am weitesten verbreiteste Technologie REST wird dies schnell sehr aufwändig. Den hier treten häufig Probleme wie Over- und Underfetching auf. Auch Facebook hatte damit zu kämpfen und entschied sich daraufhin eine neue Technologie zu entwickeln - GraphQL. Diese sollten einen flexibilen, datensparenden Zugriff auf die entsprechenden APIs ermöglichen. 

\section{Problemstellung}

Wann der Einsatz von GraphQL sinnvoll ist lässt sich aber nicht pauschal sagen. Die nachfolgende Arbeit wird verschiedene Aspekte von APIs beleuchten und untersuchen ob und wie diese in GraphQL umgesetzt werden können. 